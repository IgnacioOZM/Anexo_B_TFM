\chapter{Introducción}
\label{chap:Introducción}
\Quote{Es siempre sabio mirar adelante, pero difícil mirar más allá de lo que puedes.}{Winston Churchill (1874--1965)} 

\Abstract{Este primer capítulo introduce a la visión artificial y demuestra el gran interés otorgado por la comunidad científica en este área. También se muestra la motivación de este proyecto, así como sus objetivos y se desarrolla la estructura del mismo.}

Los orígenes de la visión artificial surgieron en 1960 como sistemas para el reconocimiento de patrones y centrados en su posible implantación en el sector industrial debido a la gran cantidad de tareas repetitivas fácilmente automatizables \citep{50years}. A pesar de la carencia de los recursos en su momento, pero debido al gran interés del mercado estos siguieron siendo desarrollados y poco a poco implantados. Una de las primeras empresas en implantar estos sistemas fue Hitachi Labs en 1964 en Japón \cite{50years}.

Desgraciadamente estos primeros desarrollos carecían de precisión y tenían una tasa de acierto del 95\% (no se consideró suficiente para su implementación en una línea de producción). Pero esto se vería resuelto en los años venideros de forma que en la década de los 70 estos sistemas pasaron a formar parte de un gran porcentaje del sector industrial. Un claro ejemplo fue la automatización de la producción de transistores con semiconductores en 1974 por Hitachi ya que, debido a su complejidad, esta tarea no podía ser llevada a cabo por humanos de forma segura \cite{hitachi}.

En la actualidad se pueden distinguir dos tipos de visión artificial en función de sus capacidades de adaptación. En primer lugar, se encuentran los sistemas  desarrollados a partir 1960 que se caracterizan por ser programas para cumplir un único objetivo bajo unas circunstancias dadas. Estos son los menos potentes ya que no se adaptan y no saben reaccionar ante un cambio, pero son los más fáciles de desarrollar. Estos sistemas se basan en características diferenciadoras del objeto de trabajo como puede ser la forma, color, un patrón... Además, en espacios controlados pueden llegar a dar mejores resultados que sistemas más complejos y avanzados \cite{ABB}. Pero es debido a su falta de flexibilidad y a las limitaciones de características diferenciadoras de las piezas de trabajo que están siendo sustituidos.

Por otro lado, con el desarrollo de las redes neuronales y de inteligencias artificiales, se están desarrollando sistemas capaces de aprender y adaptarse a la situación a base de prueba y error. Se trata de sistemas muy modernos todavía en desarrollo que prometen traer avances como la conducción autónoma, detección de emociones en humanos, etc. Estos sistemas necesitan de grandes cantidades de bases de datos de las que aprender. Y sobre las que generar sus propias conclusiones y reglas de conducta. Su nivel de adaptabilidad y flexibilidad se ve definido por las capacidades de aprendizaje del sistema (neuronas y estructura de estas) así como de la riqueza de la base de datos. Pero gracias a los últimos avanzes, han conseguido superar la barrera de tecnología en desarrollo y se han empezado a implantar en sistemas reales.

El fin de este proyecto es la generación de un nuevo sistema de visión artificial para el reconocimiento de piezas de uso industrial. Este sistema se implantará dentro de una cadena de suministro del Grupo Antolín\textsuperscript{\textregistered} que a su vez alimentará a una linea de montaje y ensamblaje. El sistema deberá de poder identificar múltiples piezas y determinar el punto de agarre óptimo de cada pieza. Este se ve definido por sus coordenadas así como el vector normal a la superficie. De esta forma un brazo robótico con un sistema de agarre por aspiración, ventosa o \textit{soft-robotics} (varía en función de la pieza a coger) podrá recolectarlas. La multitud de herramientas de agarre así como de la necesidad de un sistema de determinación de puntos de agarre se debe a la gran variedad de piezas existentes. Estas presentan una gran variedad de tamaños (1-30 cm) y diversas formas.

Como se ha mencionado en el párrafo anterior, las capacidades de una red neuronal se ven limitadas tanto por la estructura de la red como por la riqueza de la base de datos. Y para esta aplicación concreta la obtención de dicha base de datos presenta un gran desafío debido a la gran cantidad de información que se requiere de cada imagen (identificación de las piezas, posibles puntos de agarre de cada pieza y el vector normal a dichos puntos). Es por lo que se ha optado por desarrollar a su vez una base de datos sintética que permita generar todas las imágenes e información necesarias. De esta forma se podrá automatizar el proceso de aprendizaje de la red para la introducción de nuevas piezas.

\section{Motivación}
\label{chap:Introducción sec:Motivación}
La industria avanza a un ritmo constante y cada día es más necesario la implantación de sistemas robóticos para poder llevar a cabo tareas que los humanos no podemos desarrollar o que presentan una elevada tasa de errores humanos o tienen un elevado nivel de peligrosidad. Al dotar a estos sistemas de inteligencia y un sistema de visión artificial, se consigue que se pueda adaptar mejor al entorno y se evite tener que re-programar los robots con cada cambio de las condiciones de operación. Avanzamos hacia una sociedad en la que los robots serán la principal mano de obra y para llegar a ese objetivo es necesario invertir y desarrollar más los sistemas actuales. Por ello se ha propuesto como proyecto la integración de un sistema de visión artificial en un robot industrial. Con este proyecto se podrá modernizar las instalaciones del Grupo Antolín\textsuperscript{\textregistered} y aumentar las capacidades de la línea de montaje y ensamblaje actual.

\section{Objetivos}
\label{chap:Introducción sec:Objetivos}
Este trabajo parte de un proyecto actualmente en desarrollo con un sistema de visión artificial ya desarrollado e implantado. Desgraciadamente la implantación actual presenta limitaciones así como una tasa de error superior a los requisitos planteados por el Grupo Antolín.

Para poder superar las limitaciones actuales se ha planteado el desarrollo de un nuevo sistema que debe de mejorar las capacidades de identificación y detección de las piezas así como poder determinar el punto de agarre óptimo. El sistema debe de ser modular de forma que se pueda adaptar fácilmente a nuevas piezas. Y con el fin de que el robot tenga la mayor posibilidad de coger la pieza, también se debe de determinar el vector normal al punto de agarre. Por último, se pide que sea independiente del resto del proyecto de forma que pueda ser fácilmente manejado, modificado y actualizado.

Con el fin de cumplir dichos requisitos se deben plantear unos objetivos de corto, medio y largo alcance a seguir durante todo el desarrollo del trabajo:

\begin{itemize}
	\item Desarrollo de una base de datos sintética:
	\begin{itemize}
		\item Investigación y prueba de diferentes sistemas/herramientas (bpi, zpi Zumo labs y Blenderproc).
		\item Obtención de primeras imágenes RGB y de profundidad.
%		\item Obtención de imágenes de profundidad, mapa de normales, identificación de las piezas, mapas de segmentación, etc.
		\item Introducción de aleatoriedad y repetibilidad al sistema.
		\item Generación de un \textit{Pipeline} para la automatización del proceso.
		\item Introducción de ruido en las imágenes para mejorar la robustez.
	\end{itemize}
	\item Desarrollo de una base de datos real complementaria tanto para la fase de entrenamiento como evaluación del sistema.
	\item Desarrollo de las nuevas redes neuronales:
	\begin{itemize}
		\item Investigación y pruebas de diferentes sistemas/herramientas (YOLO, MobilNet, redes regresoras, etc).
		\item Planteamiento y definición de la estructura modular a desarrollar. Una primera red para la detección de piezas, seguida de una red para la detección de zonas de intereses con probabilidad de ser el punto de agarre óptimo. Y finalmente un una red neuronal del tipo regresor para la determinación del punto de agarre. 
		\item Primeros desarrollos independientes de las partes modulares del sistema
		\item Desarrollo y corroboración del conjunto modular
		\item Creación del sistema de evaluación tanto del conjunto como de las partes modulares del sistema
		\item Entrenamiento y evaluación de todo el sistema con diferentes configuraciones y bases de datos.
		\item Comparativa y determinación del sistema óptimo.
		\item Implantación del nuevo sistema de visión artificial
	\end{itemize}
\end{itemize}